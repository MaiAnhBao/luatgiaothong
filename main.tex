\documentclass{article}
\usepackage[utf8]{inputenc}
\usepackage[utf8]{vietnam}

\title{Luật giao thông}
\author{chuyendungdangki@yahoo.com}
\date{September 2015}

\begin{document}
\noindent
\parindent0pt
\maketitle

\section{Mẹo trả lời câu hỏi}

Nhóm xe ưu tiên: 

\begin{itemize}
\item[Nhóm A] Xe chữa cháy trên đường làm nhiệm vụ
\item[Nhóm B] Xe quân sự, xe công an
\item[Nhóm C] Xe cứu thương
\item[Nhóm D] Xe hộ đê khắc phục thiên tai
\item[Nhóm E] Đoàn xe tang
\end{itemize}

Quyền được đi trước qua nơi giao nhau

\begin{itemize}
\item Ở nơi giao nhau không có báo hiệu vòng xuyến, nhường đường cho xe bên tay phải
\item Ở nơi giao nhau có báo hiệu vòng xuyến, nhường đường cho xe bên tay trái
\end{itemize}

Các cách trả lời câu hỏi:

\begin{itemize}
\item Vuông xuôi đi trước, lộn ngược đi sau
\item Câu hỏi có ''hành vi``, chọn đáp án ''bị nghiêm cấm``, nếu không có đáp án ''bị nghiêm cấm``
\begin{itemize}
\item Nếu có hai đáp án: cả hai đều đúng
\item Nếu có nhiều hơn hai đáp án, có hai đáp án đúng
\end{itemize}
\item Đáp án ''bị nghiêm caasm`` và ''không được pheps`` là đáp án đúng.
\item Người điểu khiển giao thông > đèn báo hiệu > biển tạm thời > biển cố định
\item Thô sơ đi bên trái, cơ giới đi bên phải.
\item Khi chuyển hướng: báo hiệu trước 30m, nhường đường cho 4 đối tượng.
\item Trong khu vực đông dân cư, 3 bánh-30km/h; 2 bánh-40km/h; hạng B2-50km/h; 
\item Câu hỏi đạo đức, đáp án 1 và 2 là đáp án đúng.
\item Tốc đọ tương ứng với các loại phương tiện
\begin{itemize}
\item 80km/h - đáp án 1
\item 70km/h - đáp án 2
\item 50km/h - đáp án 3
\item 60km/h - đáp án 4
\end{itemize}
\item Khoảng cách an toàn giữa hai xe tương ứng với tốc độ  chuyển: Tốc độ lớn hơn - 30 = khoảng cách an toàn
\item Nhỏ hơn 20kg, miễn phí cước hành lý
\item Âm lượng còi: lớn nhất
\item Tăng số- đáp án 1, giảm số - đáp án 2
\end{itemize}

\section{54 mẹo trả lời câu hỏi}

\begin{itemize}
\item Những hành vi Cấm thì chọn đáp án tất cả.
\item     Kinh doanh vận tải thì chọn đán án tất cả.
\item     Đạo đức thì chọn tất cả.
\item     Câu hỏi về tốc độ trên đường cao tốc thì lấy tốc độ cao nhất trong câu hỏi trừ cho 30 thì ra được câu trả lời đúng.
\item     Ngoài khu dân cư tốc độ trên đường là 80 km/h < 3,5 tấn( câu 1).
\item     Ngoài khu dân cư tốc độ 70 km/h > 3,5 tấn(câu 2)
\item     Ngoài khu dân cư tốc độ 60 km/h = mô tô (câu 4)Ngoài khu dân cư tốc độ 50 km/h = xe gắn máy( câu 3)
\item     9.Trong khu dân cư tốc độ 50 km/h < 3,5 tấn.
\item     Trong khu dân cư tốc độ 40 km/h = mô tô, gắn máy.
\item     Trong khu dân cư tốc độ 30 km/h = xe công nông.
\item     Cảnh sát đưa tay thẳng lên thì tất cả phương tiện phải dừng lại.
\item     Cảnh sát đưa hai tay hoặc 1 tay giang ngang thì những xe trước và sau phải dừng lại.
\item     16 tuổi – xe gắn máy dưới 50 cm3.
\item     18 tuổi – hạng A1, A2, B2.
\item     21 tuổi – hạng C
\item     24 tuổi – hạng D
\item     27 tuổi – hạng E. (Lưu ý với giấy phép lái xe từ hạng B2 đến hạng E cách nhau 3 tuổi, đó là mẹo học luật lái xe ôtô dễ nhớ nhất)
\item     Giấy phép hạng FE được điều khiển xe có kéo rơ moóc, ô tô chở khách nối toa và không được điều khiển ô tô đầu kéo sơ mi rơ moóc.( lưu ý nếu gặp câu hỏi này thì FE = câu 1).
\item     Giấy phép hạng FC được điều khiển xe có kéo rơ moóc, ô tô đầu kéo sơ mi rơ moóc và không được điều khiển ô tô chở khách hàng nói toa, mô tô hai bánh.( lưu ý nếu gặp câu hỏi này thì FC = câu 2)
\item     Quá tải, quá khổ và vận chuyển hàng nguy hiểm thì cơ quan có thẩm quyền cấp phép.
\item     Cấm đi, cấm đỗ, cấm dừng, đường ngược chiều … thì UBND cấp tỉnh quán lý.
    \item Xe chở người và hàng nguy hiểm thì chính phủ qui định.
\item     Khái niệm xe tải trọng là xe có tải trọng trục xe vượt quá năng lực chịu tải của mặt đường.
\item     Khái niệm “phương tiện giao thông cơ giới đường bộ” chọn ( kể cả xe máy điện).
\item     Khái niệm “phương tiện giao thông thô sơ đường bộ” chọn (kể cả xe đạp máy).
\item     Khái niệm “làn đường” chọn câu có chữ “an toàn giao thông”.
\item     Khái niệm “phần đường xe chạy” chọn câu không có chứ “ an toàn giao thông”
\item     Khái niệm “làn đường” chọn câu có chữ “an toàn giao thông”
\item     Hãy nêu yêu cầu của kính chắn gió chọn câu có chữ “loại kính an toàn”
\item     Niên hạn sử dụng của ôtô tải = 25 năm.
\item     Niên hạn sử dụng của ô tô trên 9 chỗ = 20 năm.
\item     Những câu hỏi về “mét” thì chọn câu 1.
\item     Những câu hỏi về “tuổi thì chọn câu 2 (lưu ý: không áp dụng cho những câu trả lời có từ “Tuổi”.
\item     Những câu hỏi về “gương. Còi, khoảng cách” thì chọn câu 1 nhé.
\item     Giao nhau có biển báo vòng xuyến thì nhường cho xe đi bên phải.
\item     Đỗ xe: không giới hạn thời gian, dừng xe: có giới hạn thời gian –> tất cả.
\item     Nồng độ cồn trong máu = 50; nồng độ cồn trong khí thở = 0.25.
\item     Công dụng của hệ thống lái thì chịn không có chữ “mô men”.
\item     Mục đích của bảo dưỡng thường xuyến -> giữ được hình thức bên ngoài.
\item     Nguyên nhân xăng không vào buồng phao của bộ chế hòa khí -> tắc bầu lọc.
\item     Phương pháp khắc phục giclơ bị tắc -> thông lỗ giclơ bằng khí nén.
\item     Nguyên nhân thông thường động cơ điezen không nổ -> không có tia lửa điện.
\item     Động cơ 2 kì -> thực hiện 2 hành trình.
\item     Động cơ 3 kì -> thực hiện 4 hành trình.
\item     Phương pháp điều chỉnh lửa sớm sang muộn -> cùng chiều với bộ cam.
\item     Phương pháp điều chỉnh lửa muộn sang sớm -> ngược chiều với bộ cam.
\item     Độ rơ vnàh tay lái của vô lăng cho phép đối với xe con = 10 độ.
\item     Độ rơ vành tay lái của vô lăng cho phép đối với xe khách = 20 độ.
\item     Độ rơ vành tay lái của vô lăng cho phép đối với xe tải = 25 độ.
\item     Biển báo hiệu lệnh đặt trước ngã ba, ngã tư nếu câu hỏi 1 dòng thì chọn câu 1 và câu hỏi 2 dòng thì chọn câu 3.
\item     Biển báo cấm máy kéo thì không cấm ô tô tải và cấm ô tô tải thì cấm máy kéo.
\item     Biển báo câm rẽ trái thì cấm quay đầu và cấm quay đầu thì không cấm rẽ trái.
\item     Trong sa hình nếu thấy xuất hiện công an thì chọn đáp an là câu 3.
\end{itemize}
\end{document}
