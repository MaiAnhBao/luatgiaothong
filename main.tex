\documentclass{article}
\usepackage[utf8]{inputenc}
\usepackage[utf8]{vietnam}

\title{Luật giao thông}
\author{chuyendungdangki@yahoo.com}
\date{September 2015}

\begin{document}
\noindent
\parindent0pt
\maketitle

\section{Mẹo trả lời câu hỏi}

Nhóm xe ưu tiên: 

\begin{itemize}
\item[Nhóm A] Xe chữa cháy trên đường làm nhiệm vụ
\item[Nhóm B] Xe quân sự, xe công an
\item[Nhóm C] Xe cứu thương
\item[Nhóm D] Xe hộ đê khắc phục thiên tai
\item[Nhóm E] Đoàn xe tang
\end{itemize}

Quyền được đi trước qua nơi giao nhau

\begin{itemize}
\item Ở nơi giao nhau không có báo hiệu vòng xuyến, nhường đường cho xe bên tay phải
\item Ở nơi giao nhau có báo hiệu vòng xuyến, nhường đường cho xe bên tay trái
\end{itemize}

Các cách trả lời câu hỏi:

\begin{itemize}
\item Vuông xuôi đi trước, lộn ngược đi sau
\item Câu hỏi có ''hành vi``, chọn đáp án ''bị nghiêm cấm``, nếu không có đáp án ''bị nghiêm cấm``
\begin{itemize}
\item Nếu có hai đáp án: cả hai đều đúng
\item Nếu có nhiều hơn hai đáp án, có hai đáp án đúng
\end{itemize}
\item Đáp án ''bị nghiêm caasm`` và ''không được pheps`` là đáp án đúng.
\item Người điểu khiển giao thông > đèn báo hiệu > biển tạm thời > biển cố định
\item Thô sơ đi bên trái, cơ giới đi bên phải.
\item Khi chuyển hướng: báo hiệu trước 30m, nhường đường cho 4 đối tượng.
\item Trong khu vực đông dân cư, 3 bánh-30km/h; 2 bánh-40km/h; hạng B2-50km/h; 
\item Câu hỏi đạo đức, đáp án 1 và 2 là đáp án đúng.
\item Tốc đọ tương ứng với các loại phương tiện
\begin{itemize}
\item 80km/h - đáp án 1
\item 70km/h - đáp án 2
\item 50km/h - đáp án 3
\item 60km/h - đáp án 4
\end{itemize}
\item Khoảng cách an toàn giữa hai xe tương ứng với tốc độ  chuyển: Tốc độ lớn hơn - 30 = khoảng cách an toàn
\item Nhỏ hơn 20kg, miễn phí cước hành lý
\item Âm lượng còi: lớn nhất
\item Tăng số- đáp án 1, giảm số - đáp án 2
\end{itemize}
\end{document}
